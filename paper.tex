\documentclass[12pt, a4paper, twoside]{article}
\usepackage{indentfirst}
\usepackage[backend=biber, citestyle=ieee]{biblatex}
\usepackage{ragged2e}
\addbibresource{references.bib}
\title{Klasifikasi Kualitas Tidur Mahasiswa Berdasarkan Nilai Semester Satu Sampai Semester Lima Menggunakan Neural Network}
\author{Natan Hari Pamungkas}
\date{}

\begin{document}
\maketitle

\begin{abstract}
  %Todo = Write an abstract
\end{abstract}

Keywords: Data Mining, Neural Network, Sleep Quality

\section{Pendahuluan}
\justifying
Nilai mata kuliah adalah komponen penting bagi mahasiswa. Sebagian mahasiswa biasanya berusaha untuk menjaga nilai mata kuliah mereka agar minimal lulus sehingga tidak perlu mengulang. Nilai mata kuliah tiap semester ini juga yang akan mempengaruhi Indeks Prestasi Kumulatif (IPK) mahasiswa, dimana IPK akan digunakan untuk seleksi melamar pekerjaan. Meskipun bukan jaminan untuk mendapatkan pekerjaan, tapi IPK digunakan sebagai syarat administratif yang mutlak, sehingga jika IPK
kurang maka kemungkinan besar akan gagal untuk mendapatkan pekerjaan. \cite{Idris.2020}

Demi mengejar nilai yang baik, beberapa mahasiswa memiliki kebiasaan begadang. Tugas yang banyak membuat mahasiswa perlu mengorbankan waktu tidur untuk dapat menyelesaikannya. Semakin banyak tugas yang dapat diselesaikan dengan baik maka nilai mahasiswa juga akan semakin baik. \cite{Nielton.2019}

Kebiasan begadang demi meningkatkan nilai mata kuliah membuat mahasiswa mengalami kekurangan jam tidur. Hal ini disebabkan karena kebanyakan memiliki aktivitas di pagi sampai malam hari dan masih dilanjutkan untuk mengerjakan tugas sampai pagi hari lagi. Karena harus mengorbankan jam tidur untuk mengerjakan tugas, maka sebagian mahasiswa tidak dapat memenuhi kebutuhan jam tidur yang direkomendasikan, yaitu untuk usia 18-25 tahun adalah 7-9 jam. \cite{Hirshkowitz.2015}

Secara kesehatan, kekurangan jam tidur dapat menimbulkan masalah serius. Masalah-masalah tersebut dapat mempengaruhi baik fisik maupun mental seseorang. Masalah fisik yang dapat diderita oleh orang yang mengalami kekurangan jam tidur diantaranya adalah diabetes, kanker, dan penyakit jantung. Sedangkan masalah kesehatan mental yang dapat diderita diantaranya adalah \textit{anxiety}, depresi, dan insomnia. \cite{Berglund.2019}

\printbibliography[title = Daftar Pustaka]
\end{document}
