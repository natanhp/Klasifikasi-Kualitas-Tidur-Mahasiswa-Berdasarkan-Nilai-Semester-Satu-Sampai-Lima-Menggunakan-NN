\documentclass[12pt, a4paper, twoside]{article}
\usepackage{indentfirst}
\usepackage[backend=biber, citestyle=ieee]{biblatex}
\usepackage{ragged2e}
\addbibresource{references.bib}
\title{Klasifikasi Kualitas Tidur Mahasiswa Berdasarkan Nilai Semester Satu Sampai Semester Lima Menggunakan Neural Network}
\author{Natan Hari Pamungkas}
\date{}

\begin{document}
\maketitle

\begin{abstract}
  %Todo = Write an abstract
\end{abstract}

Keywords: Data Mining, Neural Network, Sleep Quality

\section{Pendahuluan}
\justifying
Nilai mata kuliah adalah komponen penting bagi mahasiswa. Sebagian mahasiswa biasanya berusaha untuk menjaga nilai mata kuliah mereka agar minimal lulus sehingga tidak perlu mengulang. Nilai mata kuliah tiap semester ini juga yang akan mempengaruhi Indeks Prestasi Kumulatif (IPK) mahasiswa, dimana IPK akan digunakan untuk seleksi melamar pekerjaan. Meskipun bukan jaminan untuk mendapatkan pekerjaan, tapi IPK digunakan sebagai syarat administratif yang mutlak, sehingga jika IPK
kurang maka kemungkinan besar akan gagal untuk mendapatkan pekerjaan. \cite{Idris.2020}

Demi mengejar nilai yang baik, beberapa mahasiswa memiliki kebiasaan begadang. Tugas yang banyak membuat mahasiswa perlu mengorbankan waktu tidur untuk dapat menyelesaikannya. Semakin banyak tugas yang dapat diselesaikan dengan baik maka nilai mahasiswa juga akan semakin baik. \cite{Nielton.2019}

Kebiasan begadang demi meningkatkan nilai mata kuliah membuat mahasiswa mengalami kekurangan jam tidur. Hal ini disebabkan karena kebanyakan memiliki aktivitas di pagi sampai malam hari dan masih dilanjutkan untuk mengerjakan tugas sampai pagi hari lagi. Karena harus mengorbankan jam tidur untuk mengerjakan tugas, maka sebagian mahasiswa tidak dapat memenuhi kebutuhan jam tidur yang direkomendasikan, yaitu untuk usia 18-25 tahun adalah 7-9 jam. \cite{Hirshkowitz.2015}

Secara kesehatan, kekurangan jam tidur dapat menimbulkan masalah serius. Masalah-masalah tersebut dapat mempengaruhi baik fisik maupun mental seseorang. Masalah fisik yang dapat diderita oleh orang yang mengalami kekurangan jam tidur diantaranya adalah diabetes, kanker, dan penyakit jantung. Sedangkan masalah kesehatan mental yang dapat diderita diantaranya adalah \textit{anxiety}, depresi, dan insomnia. \cite{Berglund.2019}

Pada saat artikel ini ditulis, Indonesia sedang dalam perlawanan untuk melawan pandemi COVID-19. Penyakit yang disebabkan oleh virus SARS-CoV-2 ini pada tanggal 28 Mei 2020 telah menambah catatan jumlah pasien yang positif terinfeksi sebanyak 687 orang. Hal ini menandakan bahwa Indonesia masih dalam keadaan belum aman. \cite{Putri.2020} 

Hasil penelitian telah menemukan bahwa adanya korelasi antara obesitas dan COVID-19. Orang dengan obesitas memiliki resiko lebih tinggi untuk menderita pneumonia akut sebagai penyakit sekunder saat terinfeksi COVID-19. Hal ini membuktikan bahwa obesitas lebih rentan untuk menderita COVID-19 lebih parah dari orang dengan berat yang normal. \cite{Cai.2020}

Orang yang mengalami kekurangan jam tidur memiliki risiko yang tinggi untuk menderita obesitas. Hal ini disebabkan karena kekurangan tidur membuat tubuh seseorang mengalami hal-hal seperti resistensi terhadap hormon insulin sehingga tidak dapat mengatur kadar gula darah dengan baik, tidak dapat mengontrol nafsu makan karena disregulasi neuroendokrin, dan kurangnya aktifitas yang membakar energi. Hal ini tentu saja sangat berbahaya karena selain rentan terinfeksi COVID-19, orang
yang menderita obesitas juga memiliki risiko tinggi untuk menderita penyakit lain. \cite{Knutson.2007}

Selain masalah kesehatan, kekurangan jam tidur dapat mengakibatkan penurunan performa bekerja pada seseorang. Kehilangan jam tidur selama 17-19 jam memiliki efek yang sama saat seseorang mengkonsumsi alkohol hingga memiliki BAC (\textit{Blood Alcohol Content}) sebanyak kurang lebih 0,05 \%. Penurunan performa tentu saja membuat mahasiswa tidak dapat belajar di kelas dengan maksimal. \cite{Williamson.2000}

\printbibliography[title = Daftar Pustaka]
\end{document}
